\documentclass[PhD]{PHlab-thesis}
\reversemarginpar        % 避免錯誤;不關心邊註方向時可保留
\setlength{\marginparwidth}{0pt}   % 邊註寬度設 0
\setlength{\marginparsep}{0pt}     % 邊註與正文的間距也設 0
\addbibresource{thesis.bib}

\newcommand*\Department中文{資訊工程學系}
\newcommand*\Department英文{Department of Computer Science and Information Engineering}

\newcommand*\ThesisTitle中文{WebGPU 於高效能生物資訊計算:瀏覽器端 Pair-HMM Forward 演算法之優化}
\newcommand*\ThesisTitle英文{ WebGPU for High-Performance Bioinformatics Computing: Browser-Based Optimization of the Pair-HMM Forward Algorithm}

\newcommand*\Student中文{王尊緯}
\newcommand*\Student英文{Tsun-Wai Wang}

\newcommand*\Advisor中文{賀保羅}
\newcommand*\Advisor英文{Paul Horton}

%% 果有共同指導老師可以用:
%% \newcommand*\CoAdvisorA中文{}
%% \newcommand*\CoAdvisorA英文{}
%% \newcommand*\CoAdvisorB中文{}
%% \newcommand*\CoAdvisorB英文{}


\newcommand*\YearMonth英文{July, 2025}
\newcommand*\YearMonth中文{114年7月}

\pagestyle{fancy}% Use fancyhdr

\usepackage{fancyhdr}
\pagestyle{fancy}
\setlength{\headheight}{13.6pt}
\renewcommand{\chaptermark}[1]{\markboth{Chapter \thechapter.\ #1}{}}
\fancyhead{}

\fancyhead[C]{\small\MakeUppercase{\leftmark}}


\begin{document}


% ---------- Keywords ----------
\newcommand*\Keywords英文{WebGPU, Pair-Hidden Markov Model, Bioinformatics Acceleration}
\newcommand*\Keywords中文{瀏覽器 GPU 計算、Pair-Hidden Markov Model (Pair-HMM)、生物資訊加速}

% ---------- Abstract (English) ----------
% ---------- Abstract (English) ----------
\newcommand*\Abstract英文{%
As GPU acceleration becomes increasingly prevalent in bioinformatics (Banerjee \emph{et al}., 2017; Liu \emph{et al}., 2021), conventional CUDA/OpenCL solutions still require driver installation and are locked to specific hardware, hindering online teaching and front-end clinical analysis. Ratified in 2024, \textbf{WebGPU} unifies Vulkan, Direct3D 12, and Metal behind a single JavaScript API (W3C, 2024), offering three decisive advantages: zero installation, cross-hardware portability, and on-device data residency. Using the compute-intensive Pair-Hidden Markov Model Forward (Pair-HMM Forward) algorithm (Durbin \emph{et al}., 1998) as a case study, we assess the performance and viability of this new framework.

Starting from Chou, Yu-Chen’s (2024) open-source C++/CUDA implementation (Chou, 2024), we first develop a WebGPU baseline version. We then tackle its primary bottlenecks—frequent CPU$\leftrightarrow$GPU round-trips and costly BindGroup reconstruction—by successively introducing: (i) single-\emph{CommandBuffer} batch submission and (ii) Dynamic Uniform Offsets. This leads to an optimized implementation referred to as \emph{WebGPU-Optimized}.

Across an NVIDIA RTX 2070 Super, Apple M1, and Intel UHD 620, and for sequence lengths from $10^{2}$ to $10^{5}$, the optimized version achieves notable speed-ups (exceeding $100\times$ in the best case) and attains over 80 \% of CUDA’s throughput. The relative log-likelihood error remains below $10^{-5}$ on all devices. Even without access to an NVIDIA GPU, our method still outperforms single-threaded C++ by one to two orders of magnitude.

These results demonstrate that pure JavaScript and WGSL can compute Pair-HMM Forward within seconds in a browser. We contribute two browser-specific optimization strategies and detailed cross-hardware measurements, laying the groundwork for Web-native genomic analysis tools and supporting the democratization and real-time execution of bioinformatics workloads.
}

% ---------- Abstract (Chinese) ----------
\newcommand*\Abstract中文{%
隨著 GPU 加速在生物資訊領域日漸普及(Banerjee 等,2017;Liu 等,2021),傳統 CUDA/OpenCL 解決方案必須安裝驅動且受限於特定硬體,對線上教學與臨床前端分析造成不便。2024 年正式標準化的 WebGPU 以單一 JavaScript API 對接 Vulkan/D3D12/Metal(W3C,2024),兼具「免安裝、跨硬體、資料留在本機」三大優勢。本研究以高強度 Pair-Hidden Markov Model Forward(Pair-HMM Forward)演算法(Durbin 等,1998)為例,評估 WebGPU 的效能與可行性。

我們以周育晨(2024)公開之 C++/CUDA 程式為基準(Chou,2024),首先撰寫 WebGPU Baseline;接著針對 CPU↔GPU 往返與 BindGroup 重建等瓶頸,依序導入「單一 CommandBuffer 批次提交」與「Dynamic Uniform Offset」,形成 \emph{WebGPU-Optimized}。於 NVIDIA RTX 2070 Super、Apple M1 與 Intel UHD 620 針對序列長度 $10^{2}$--$10^{5}$ 之測試顯示,Optimized 相較 Baseline 能顯著加速(最高逾百倍),執行速度可達 CUDA 的八成以上;三款裝置的 log-likelihood 相對誤差皆低於 $10^{-5}$,且在無 NVIDIA GPU 時仍對單執行緒 C++ 提供數十至數百倍的加速。

本研究證實僅憑 JavaScript 與 WGSL,即能於瀏覽器中在秒級完成 Pair-HMM Forward 計算,並提出兩項瀏覽器端專屬優化策略及跨硬體實測結果。此成果為 Web-native 基因體分析工具的普及奠定基礎,推動生物資訊運算的民主化與即時化。
}



\newcommand*\Acknowledgements{%
在此特別感謝賀保羅教授以及在實驗室給予我協助與指導的同學:阮祈翰、楊祐昇、黃書堯、鄭驊軒、鄭煜醴等,在各階段實驗設計與資料收集上提供許多寶貴意見。

同時感謝林宜靜、賴威達、王晴文、許庸袁及陳竑曄,你們在討論分析與程式測試時的協作與支持,讓本研究得以順利完成。

最後,感謝所有關心及協助本研究的師長與同窗,謹此致上由衷謝意。 }



\input{frontmatter}% 封面頁, 口委中英文簽名單, 誌謝, 中英文摘要, 論文目錄, 圖表目錄


%────────────────────  List of Symbols  ────────────────────
\renewcommand\nomgroup[1]{%
  \item[\bfseries
  \ifstrequal{#1}{A}{General}{%
  \ifstrequal{#1}{Z}{Gene/Protein Names}%
  }]}

\nomenclature[A]{NGS}{Next-Generation Sequencing}
\nomenclature[A]{DP}{Dynamic Programming}
\nomenclature[A]{$LL$}{Log-Likelihood}
\nomenclature[B]{Pair-HMM}{Pair-Hidden Markov Model}
\nomenclature[B]{Wavefront}{Anti-Diagonal Processing Order}
\nomenclature[B]{$M_{i,j}$}{Match state DP value at $(i,j)$}
\nomenclature[B]{$I_{i,j}$}{Insert state DP value at $(i,j)$}
\nomenclature[B]{$D_{i,j}$}{Delete state DP value at $(i,j)$}
\nomenclature[B]{$N$}{Read length}
\nomenclature[B]{$M$}{Reference length}
\nomenclature[C]{WebGPU}{Web Graphics Processing Unit API}
\nomenclature[C]{WGSL}{WebGPU Shading Language}
\nomenclature[C]{BindGroup}{Resource binding group in WebGPU}
\nomenclature[C]{Dynamic Uniform Offset}{Per-dispatch uniform buffering offset}
\nomenclature[C]{SFU}{Special Function Unit}
\nomenclature[C]{RTX 2070 S}{NVIDIA RTX 2070 Super GPU}

\printnomenclature[5cm]

\newpage
\setcounter{page}{1}
\pagenumbering{arabic}

\chapter{Introduction}
\section{Background}
High-throughput sequencing (Next-Generation Sequencing, NGS) has pushed the scale and complexity of genomic data to grow exponentially (Mardis, 2017), creating unprecedented demands on computational performance in bioinformatics. The \textbf{Pair Hidden Markov Model (Pair-HMM) Forward algorithm}, which simultaneously supports sequence alignment, genotype calling, and variant detection, is a core computation in many genomic pipelines (Durbin \emph{et al}., 1998).

Current analysis workflows—such as GATK (McKenna \emph{et al}., 2010), Samtools (Li \emph{et al}., 2009), and BWA-MEM2 (Vasimuddin \emph{et al}., 2019)—are mostly written in C++ or Python and gain GPU acceleration through NVIDIA CUDA or OpenCL (Liu \emph{et al}., 2021; Banerjee \emph{et al}., 2017). Besides installing drivers, SDKs, and dependencies, users are tied to specific GPU architectures. In classrooms or resource-constrained labs, the lack of high-end GPUs or sufficient cloud quotas often forces a CPU fallback, dramatically inflating cost and turnaround time. Cloud services reduce local setup complexity but introduce account management, network latency, and concerns over sensitive-data exposure.

Since WebGPU landed in Chrome's stable channel in May 2024 (Google Chrome Developers, 2024), Firefox Nightly and Edge Dev have followed with experimental support. By mapping Vulkan, Direct3D 12, and Metal to a single JavaScript API inside the browser sandbox (W3C, 2024), WebGPU enables real-time parallel computation on NVIDIA, AMD, Intel, and Apple Silicon GPUs without requiring any driver installation. It therefore offers a new opportunity to lower the barrier to bioinformatics tools, especially in teaching, clinical front-ends, and low-resource settings.

\section{Motivation and Objectives}
Although WebGPU promises zero installation, cross-hardware portability, and on-device data residency, it was designed chiefly for graphics and ML inference. A compute-intensive dynamic-programming algorithm such as Pair-HMM Forward encounters several challenges:
\begin{itemize}
    \item \emph{API scheduling overhead} – Wavefronts must be processed sequentially; if each dispatch creates fresh CommandBuffers and BindGroups, CPU$\leftrightarrow$GPU round-trips accumulate rapidly.
    \item \emph{Lack of global synchronization} – Every wavefront depends on the previous one; WebGPU exposes only workgroup-level barriers and lacks CUDA-style kernel-return global sync (W3C, 2024).
    \item \emph{No dedicated special-function units (SFUs)} – Pair-HMM issues many \texttt{log}/\texttt{exp} calls. CUDA's SFUs finish \texttt{log₂} in four cycles (NVIDIA, 2023), whereas WebGPU must approximate via ALU + LUT + FMA, adding a 2–4$\times$ latency penalty.
    \item \emph{High memory-access demand} – The DP matrix resides in a read-write storage buffer; WebGPU's default path bypasses L1, so frequent global reads and writes incur heavy DRAM traffic (Liu \emph{et al}., 2021).
\end{itemize}
These factors magnify WebGPU's still-maturing API and hardware limits, and no systematic study yet verifies its suitability for high-intensity bioinformatics workloads. We therefore ask: \emph{Can WebGPU inside a browser execute Pair-HMM Forward with sufficient performance to serve as a practical alternative to CUDA?}

\section{Methods and Key Results}
Using Chou, Yu-Chen's (2024) open-source C++/CUDA program as the reference (Chou, 2024), we introduce two WebGPU-specific optimizations to address the above bottlenecks:
\begin{enumerate}
    \item \emph{Single-CommandBuffer batch submission} – Aggregate multiple wavefronts into one \texttt{queue.submit()}, eliminating extensive API round-trip latency.
    \item \emph{Dynamic Uniform Offsets} – Store static parameters in a single uniform buffer and access them via dynamic offsets, removing the need to allocate and bind multiple UBOs.
\end{enumerate}
On an NVIDIA RTX 2070 Super with sequence lengths 100–100 000, “WebGPU-Optimized” accelerates the Baseline by 6.8–142$\times$ and reaches 11–84 \% of CUDA's throughput. On Apple M1 and Intel UHD 620, it delivers 4–463$\times$ speed-ups over CPU execution for sequences \(\ge 1{,}000\), while maintaining log-likelihood errors below \(10^{-5}\).

\section{Conclusions and Contributions}
This study shows that JavaScript plus WGSL can solve medium-to-large Pair-HMM Forward instances within seconds in a browser sandbox. The two complementary browser-side optimizations effectively mitigate API, synchronization, and memory bottlenecks, and cross-vendor tests on NVIDIA, Apple, and Intel GPUs confirm hardware agnosticism. Our results pave the way for “open-the-browser-and-compute” genomic analysis, and they provide an empirical foundation for future work on double-precision support and WASM-SIMD + WebGPU hybrid acceleration.


\chapter{Related Work}

\section{High-Performance Computing Requirements in Bioinformatics}

\subsection{Next-Generation Sequencing and Its Computational Challenges}
The rapid advance of Next-Generation Sequencing (NGS) has driven genomic data volume to grow exponentially (Mardis, 2017). According to Illumina's specifications, a NovaSeq X Plus equipped with a 25 B flow cell in dual-lane mode can generate roughly 52 billion ($5.2 \times 10^{10}$) paired-end reads (2 $\times$ 150 bp) in a single 48-hour run—about $3.25 \times 10^{11}$ bases per hour (Illumina, 2024). Such data volumes demand massive sequence alignment and probabilistic computation well beyond what traditional CPU-only architectures can sustain.

Alignment tools such as BWA (Li \& Durbin, 2010) and Bowtie (Langmead \emph{et al}., 2009) routinely process millions to billions of short reads. While their worst-case time complexity is $O(NM)$, FM-index–based seeding and extension make the average cost nearly linear $O(L)$ in read length \emph{L} (Ferragina \& Manzini, 2000). These challenges have driven researchers to adopt high-performance computing (HPC) solutions—especially GPU acceleration—to meet the real-time demands of modern bioinformatics workflows (Liu \emph{et al}., 2021).

\subsection{The Central Role of the Pair-HMM Forward Algorithm}
The Pair-Hidden Markov Model (Pair-HMM) Forward algorithm is a key component for sequence alignment and genotype inference. It underpins tools such as GATK (McKenna \emph{et al}., 2010) and Samtools (Li \emph{et al}., 2009). As described by Durbin \emph{et al}. (1998, chap. 4) and Banerjee \emph{et al}. (2017), the algorithm has a time complexity of $O(NM)$, with memory use reducible to $O(\max\{N,M\})$. For long reads (\emph{N} ≈ $10^{4}$), Pair-HMM Forward becomes a major computational bottleneck. Recent studies show that GPU parallelisation can dramatically accelerate this step—for example, Schmidt \emph{et al}. (2024) cut the processing time for 32 × 12 kb fragments on an NVIDIA RTX 4090 from hours to under three minutes. Nonetheless, the algorithm's sensitivity to memory-access patterns and numerical precision demands hardware-aware optimisation.

\section{Conventional GPU Acceleration Frameworks: CUDA and OpenCL}

\subsection{CUDA in Bioinformatics}
NVIDIA CUDA has become the de-facto standard for GPU computing in bioinformatics. Liu \emph{et al}. (2013) developed CUDASW++ 3.0, accelerating the Smith–Waterman algorithm by 30–90 × over single-threaded CPUs. Schmidt \emph{et al}. (2024) further applied CUDA to Pair-HMM Forward, achieving high throughput in large-scale genotyping workflows. However, CUDA is tied to NVIDIA-exclusive hardware and requires driver installation plus the CUDA Toolkit, posing barriers for non-expert users. Moreover, kernels often need tuning for specific micro-architectures (e.g., Ampere, Hopper), reducing portability across devices.

\subsection{OpenCL's Cross-Platform Ambition}
OpenCL was conceived to provide hardware-agnostic GPU acceleration, supporting NVIDIA, AMD, and Intel GPUs alike. Stone \emph{et al}. (2010) showed its potential in scientific computing—for instance, accelerating molecular-dynamics simulations. Yet its use in bioinformatics remains limited compared with CUDA, chiefly due to uneven hardware support and a steeper development curve. Klöckner \emph{et al}. (2012) reported that OpenCL's memory-management and thread-synchronisation behaviour varies markedly across devices, producing inconsistent performance. In addition, the OpenCL ecosystem is less mature than CUDA's and lacks a broad library base, further constraining adoption in bioinformatics.

\subsection{Barriers and Limitations of Traditional Frameworks}
In summary, while CUDA and OpenCL offer high performance, both demand intricate setup steps—driver installation, SDK configuration, and dependency management—that impede deployment in educational settings and clinical front-ends. Cloud-based GPU services (e.g., AWS, Google Cloud) mitigate local setup but introduce data-transfer latency and privacy concerns (Krampis \emph{et al}., 2012). Furthermore, these solutions are tightly coupled to specific hardware architectures and do not provide true cross-platform compatibility, limiting their practicality on resource-constrained devices such as laptops and embedded systems.

% ---------- 5.x The Emergence and Technical Characteristics of WebGPU ----------
\section{The Emergence and Technical Characteristics of WebGPU}

\subsection{Technical Background}
On 19 December 2024, WebGPU entered the W3C Candidate Recommendation Snapshot stage; it has not yet reached full Recommendation status and therefore still awaits complete implementations and interoperability tests (W3C, 2024). Figure~\ref{fig:webgpu-mapping} illustrates WebGPU's architecture: a single JavaScript / TypeScript API translates application calls to Vulkan, Direct3D 12, or Metal back ends, which are then dispatched to the underlying GPU. This design offers three major advantages—driver-free installation, cross-platform compatibility, and browser-sandbox safety (Google Chrome Developers, 2024).

WebGPU's compute pipeline is written in WGSL (WebGPU Shading Language), which supports high-performance matrix operations and explicit memory control (W3C, 2024), making it suitable for compute-intensive workloads.

\begin{figure}[htbp]
    \centering
    \includegraphics[width=0.9\linewidth]{WebGPU 架構對應示意.png}
    \caption{Mapping between the WebGPU API and native back ends}
    \label{fig:webgpu-mapping}
\end{figure}

JavaScript/TypeScript invokes WebGPU through a unified interface. The browser selects Vulkan (Linux), Direct3D 12 (Windows), or Metal (macOS/iOS) as the actual back end and submits commands to the GPU.

\subsection{High-Performance Computing Potential of WebGPU}
WebGPU has already shown promise in graphics and machine-learning domains. MDN Web Docs (2025) reports game-rendering performance on WebGPU that rivals native Vulkan, while TensorFlow.js accelerates in-browser neural-network training via its WebGPU back end (TensorFlow.js Team, 2024). According to the Google Chrome Team (2024), Transformers.js running BERT-base on an NVIDIA RTX 4060 Laptop is 32.51$\times$ faster in WebGPU mode than in WebAssembly, underscoring WebGPU's performance potential.

These cases demonstrate that WebGPU's execution model can exploit GPU parallelism effectively. However, its adoption in bioinformatics remains nascent because the field demands stricter numerical precision and memory-access efficiency. WebGPU trades some low-level control for broad hardware portability, which distinguishes its compute shaders from those of traditional GPU frameworks.

\subsection{Challenges and Limitations of WebGPU}
Despite its cross-platform appeal, WebGPU still faces hurdles in high-intensity computing. First, browser-side API scheduling overhead is relatively high, particularly during frequent CPU–GPU data exchanges (Google Chrome Team, 2024). Second, WebGPU exposes only \texttt{workgroupBarrier} and lacks any cross-workgroup global-synchronisation primitive, hampering algorithms that require complex thread coordination (W3C, 2024).

Moreover, the browser sandbox constrains memory allocation and special-function-unit (SFU) usage. WGSL lacks built-in transcendental functions (e.g., \texttt{log}, \texttt{exp}) and must rely on software emulation; although the GPU may execute these via SFUs, additional overhead remains. Double-precision (\texttt{f64}) support is also less mature than in CUDA (Jones, 2023). The wavefront dependencies of Pair-HMM Forward demand frequent memory access and thread synchronisation, making high-performance parallelisation challenging under WebGPU's current limitations.

\section{Initial Explorations of WebGPU in Bioinformatics}
Direct WebGPU applications in bioinformatics are still scarce, but related technologies such as WebGL and WebAssembly (WASM) provide useful precedents. Ghosh \emph{et al}. (2018) used WebGL to build Web3DMol, showing that in-browser molecular visualisation is feasible and that JavaScript can serve bioinformatics needs. WASM-SIMD further boosts browser-side performance; Jones (2023) argues that combining WASM-SIMD with WebGPU could accelerate sequence processing.

Other compute-intensive domains also offer insights. For instance, the WebGPU back end of TensorFlow.js speeds up browser-based neural-network training through efficient matrix operations (TensorFlow.js Team, 2024); its memory-management and parallelisation strategies inspire our port of Pair-HMM Forward, particularly for handling frequent memory access and compute-heavy kernels. Nonetheless, existing studies focus mostly on visualisation or lightweight workloads; implementations of high-intensity algorithms such as Pair-HMM Forward remain unexplored. While Schmidt \emph{et al}. (2024) provide a CUDA baseline, they do not investigate browser-specific optimisations.

\section{Research Gap and Positioning of This Work}
The literature reveals three key shortcomings:
\begin{enumerate}
    \item Lack of systematic verification of WebGPU's performance and feasibility for compute-heavy bioinformatics tasks such as Pair-HMM Forward.
    \item Absence of browser-specific optimisation strategies addressing GPU-compute bottlenecks—CPU–GPU round-trips and BindGroup reconstruction, . For example, each \texttt{setBindGroup()} call traverses multiple layers (V8 $\rightarrow$ Blink $\rightarrow$ Dawn $\rightarrow$ Driver), incurring $\sim$5–15 $\mu$s of latency (Google Chrome Developers, 2024), which is significant for wavefront algorithms.
    \item Insufficient cross-hardware evaluations (NVIDIA, Apple, Intel) to gauge WebGPU's generality. Mainstream tools like GATK HaplotypeCaller require CUDA and thus NVIDIA drivers, limiting deployment in classrooms or resource-constrained environments; cloud solutions raise latency and privacy concerns (Krampis \emph{et al}., 2012).
\end{enumerate}
By porting Chou, Yu-Chen's (2024) CUDA implementation to WebGPU, this study proposes two browser-side optimisations—\emph{single-CommandBuffer batch submission} and \emph{Dynamic Uniform Offsets} —and validates their performance and accuracy across multiple hardware platforms. The results close the research gap in WebGPU bioinformatics applications and lay the foundation for driver-free, cross-hardware, on-device genomic analysis tools.




\chapter{Methods}

\section{Mathematical Model}
This study adopts a Pair-HMM combined with a sequence profile, following the formulation in Durbin \emph{et al}. (1998, chap.~4).

The hidden states are Match (M), Insert (I), and Delete (D), over the alphabet $\{A,C,G,T,-\}$ (Durbin \emph{et al}., 1998).

The read sequence is represented by a probability matrix
\[
P = [p_{i,a}], \quad 1 \leq i \leq m, \quad a \in \{A,C,G,T\}, \quad \sum_a p_{i,a} = 1,
\]
which gives the probability that position $i$ of the read is character $a$ (Banerjee \emph{et al}., 2017).

The haplotype sequence is a fixed string $h_1, \dots, h_n$.

The transition probabilities
\[
t_{XY}, \quad X,Y \in \{M,I,D\},
\]
and the base-emission matrix $\varepsilon_X(x,y)$ follow the configuration used in the C++/CUDA implementation published by Chou, Yu-Chen (2024).

\subsection{Profile emission probabilities}
\[
e_{i,j}^M = \sum_a p_{i,a} \varepsilon_M(a, h_j), \quad 
e_{i,j}^I = \sum_a p_{i,a} \varepsilon_I(a, -),
\]
while the Delete state always emits a gap, so its emission probability is fixed at 1 (Durbin \emph{et al}., 1998).

\section{Pair-HMM Forward Algorithm}
The Pair-HMM Forward recursion must advance along the anti-diagonals (wavefronts) of the dynamic-programming (DP) matrix—as illustrated in Figure~\ref{fig:pairhmm-wavefront}. Each wavefront can proceed only after the previous one has completed; without device-side global synchronisation, this dependency becomes a performance bottleneck on the GPU (Banerjee \emph{et al}., 2017).

\begin{enumerate}
    \item \textit{Initialisation}
    \[
    M_{0,0} = 1, \quad I_{0,0} = D_{0,0} = 0, \quad 
    M_{0,j} = I_{0,j} = 0, \quad D_{0,j} = \frac{1}{n} \quad (j > 0).
    \]
    \item \textit{Recursion} (Durbin \emph{et al}., 1998)
    \[
    M_{i,j} = e_{i,j}^M (t_{MM} M_{i-1,j-1} + t_{IM} I_{i-1,j-1} + t_{DM} D_{i-1,j-1}),
    \]
    \[
    I_{i,j} = e_{i,j}^I (t_{MI} M_{i-1,j} + t_{II} I_{i-1,j}),
    \]
    \[
    D_{i,j} = t_{MD} M_{i,j-1} + t_{DD} D_{i,j-1}.
    \]
    \item \textit{Termination}
    \[
    P = \sum_{j=1}^n (M_{m,j} + I_{m,j}).
    \]
    \item \textit{Time complexity} is $\mathcal{O}(mn)$.
\end{enumerate}

\begin{figure}[htbp]
    \centering
    \includegraphics[width=0.9\linewidth]{Pair-HMM Forward 的計算沿反對角線 .png}
    \caption{Illustration of wavefront computation in the Pair-HMM Forward algorithm}
    \label{fig:pairhmm-wavefront}
\end{figure}

Left: recursion dependencies of the three hidden states M/I/D.  
Right: yellow arrows indicate batched progression along anti-diagonals; red, green, and blue squares represent the current wavefront's M, I, and D states, respectively (Schmidt \emph{et al}., 2024).

\section{System Design and Implementation}

\subsection{C++/CUDA Version}
Building on prior work showing CUDA can efficiently implement Pair-HMM via anti-diagonal parallelisation (Banerjee \emph{et al}., 2017; Schmidt \emph{et al}., 2024), we adopted and refactored the open-source C++/CUDA code by Chou, Yu-Chen (2024). All double-precision (\texttt{double}) variables were converted to single precision (\texttt{float}) so that the CUDA results represent an upper-bound “performance ceiling” directly comparable with our WebGPU implementation. Because WebGPU currently guarantees only \texttt{f32} arithmetic (W3C, 2024), retaining \texttt{f64} on CUDA would have obscured cross-platform comparisons with precision differences. After conversion, the maximum relative error at sequence length $N = 10^{5}$ was merely $2.18 \times 10^{-1}\%$, satisfying the accuracy threshold for subsequent WebGPU validation.

We preserve the “one kernel per anti-diagonal” structure: each of the $2N$ wavefronts launches a kernel, and a \texttt{cudaDeviceSynchronize()} between adjacent wavefronts acts as a GPU-wide barrier to ensure all thread-block dependencies are fully resolved (NVIDIA, 2023). This maps the DP-recurrence dependencies on the left, above, and upper-left cells to device execution in the most straightforward manner.

Global synchronisation alone, however, cannot hide memory latency. Inside each block we therefore keep a fine-grained \texttt{\_\_syncthreads()} barrier so that every 32 threads share cached values from the previous row before advancing. For the three DP arrays $M$, $I$, and $D$, we employ a host-side “four-row pointer rotation”: four $(n\!+\!1)$-length buffers are allocated once, and the host loop rotates pointers to realise the \texttt{prev $\rightarrow$ curr $\rightarrow$ new} shift (Liu, Wirawan, \& Schmidt, 2013). Because CUDA pointers can be treated as ordinary C pointers, this scheme avoids reallocations and \texttt{memcpy} overhead, maximising effective PCIe/NVLink bandwidth.

This structure also paves the way for the WebGPU port: once inside the browser we lose mutable pointers and must replace them with either BindGroup reconstruction or Dynamic Uniform Offsets (Google Chrome Developers, 2024).




\subsection{WebGPU Baseline}

\subsubsection{From CUDA “multiple kernels” to WebGPU “multiple dispatches”}
Pair-HMM Forward advances along anti-diagonals (wavefronts); each wavefront must finish before the next can begin (Durbin \emph{et al}., 1998).

The most straightforward CUDA strategy is a host-side \texttt{for} loop that launches successive kernels and inserts a \texttt{cudaDeviceSynchronize()} between them—an approach adopted in earlier GPU Pair-HMM studies (Banerjee \emph{et al}., 2017; Schmidt \emph{et al}., 2024). Figure~\ref{fig:global-sync-diff} shows that the synchronisation point remains entirely within the GPU.

In contrast, WebGPU lacks a device-side global barrier; synchronisation must return to JavaScript and invoke a new \texttt{dispatch} (W3C, 2024).

For a read length of $N$, this inevitably triggers $2N$ CPU$\leftrightarrow$GPU round-trips (Google Chrome Developers, 2024), which becomes the first major bottleneck of the Baseline implementation.

\begin{figure}[htbp]
    \centering
    \includegraphics[width=0.9\linewidth]{4. 全域同步差異圖.png}
    \caption{Comparison of global synchronisation in CUDA and WebGPU}
    \label{fig:global-sync-diff}
\end{figure}

CUDA can insert a global barrier entirely on the GPU; WebGPU must return to the host and issue a new \texttt{dispatch} to achieve equivalent synchronisation.

After the host calls \texttt{queue.submit()} for the current wavefront, it must await \texttt{device.queue.onSubmittedWorkDone()} before updating uniforms and submitting the next dispatch.  
For $N = 10^{5}$, this results in $\sim$200{,}000 \texttt{submit → await} cycles; the synchronisation latency is fully exposed on the JavaScript thread (Google Chrome Developers, 2024).

\subsubsection{Pointer rotation versus the immutability of BindGroups}
CUDA needs only to swap three \texttt{float*} pointers between two wavefronts to rotate the roles \texttt{prev → curr → new}; the driver does not reallocate resources (NVIDIA, 2023).

By contrast, in WebGPU, each binding slot is immutable once the BindGroup is created. To let the next wavefront read from a different DP buffer, the host must call \texttt{device.createBindGroup()} again to rebind the slot to a new \texttt{GPUBuffer}.  
This call traverses multiple layers—V8, Blink, Dawn, and finally the driver—incurring an average latency of 5–15 $\mu$s per call (Google Chrome Developers, 2024).

\begin{figure}[htbp]
    \centering
    \includegraphics[width=0.9\linewidth]{3. WebGPU 多層 IPC/驗證路徑.png}
    \caption{Multi-layer IPC/validation path for binding a WebGPU storage buffer}
    \label{fig:webgpu-ipc-validation}
\end{figure}

Compared with CUDA's direct host-to-VRAM path, WebGPU must cross three additional abstraction layers, resulting in elevated delay for every \texttt{createBindGroup()} operation.

\subsubsection{The absence of shared memory and high-latency storage-buffer access}
In CUDA, the nine transition coefficients and 75 emission coefficients are preloaded into 48 KB of shared memory and reused by all threads with a latency of approximately 80 ns (NVIDIA, 2023).

Although WGSL supports \texttt{var<workgroup>}, its capacity is limited and manual copying is required (W3C, 2024).  
For correctness and simplicity, the Baseline keeps these small matrices in a storage buffer. Consequently, each cell computation requires 6–9 global reads, each incurring a latency of roughly 300 ns—significantly slower than shared-memory access (Google Chrome Developers, 2024). This forms the third major bottleneck.

\subsubsection{Interim trade-offs in the Baseline}
% (條列內容無需額外引用,保留原文)

\subsubsection{Performance profile of the Baseline}
On an NVIDIA RTX 2070 Super, the Baseline implementation requires approximately 466 s to complete when $N = 100{,}000$, nearly two orders of magnitude slower than the CUDA version on the same hardware—consistent with the overhead sources analysed above (Google Chrome Developers, 2024).

Detailed profiling attributes the delay to two primary sources: $2N$ inter-process synchronisations and $2N$ BindGroup creations. These architectural bottlenecks highlight key inefficiencies specific to WebGPU and motivate the optimisation strategies explored in the next section—namely, reducing host–GPU round-trips, minimising BindGroup churn, and caching frequently accessed data within \texttt{var<workgroup>}.



\subsection{WebGPU Optimized Version}
To eliminate the Baseline's two major bottlenecks—
\begin{enumerate}
    \item frequent host synchronizations,
    \item repeated BindGroup construction, and
\end{enumerate}
we introduce two browser-side optimizations: \emph{single-CommandBuffer batch submission} and \emph{Dynamic Uniform Offsets}. Below we first recap WebGPU's command-recording pipeline (W3C, 2024), then explain each optimization's rationale, implementation, and impact.

\subsubsection{Single-CommandBuffer Batch Submission — Reducing CPU–GPU Round-Trips}
As shown in Figure~\ref{fig:scb-batch-workflow}, the original $2N$ \texttt{dispatchWorkgroups} calls are first recorded into the same \texttt{CommandBuffer}; the host finally issues a single \texttt{queue.submit()}, eliminating over 99.99\% of IPC latency (Google Chrome Developers, 2024).

\textit{CommandEncoder and the command stream.}

A WebGPU \texttt{CommandEncoder} records \texttt{beginComputePass}, \texttt{dispatchWorkgroups}, \texttt{copyBufferToBuffer}, \texttt{end}, and related commands.  
Calling \texttt{encoder.finish()} produces a \texttt{GPUCommandBuffer}, which \texttt{device.queue.submit([commandBuffer])} sends to the GPU; the GPU then executes the entire stream without further CPU intervention (W3C, 2024).

\textit{Pain points of many small submissions.}

The Baseline maintains wavefront dependencies with a host-side \texttt{for} loop that rebuilds an encoder, submits, awaits completion, and then constructs the next encoder (Google Chrome Developers, 2024). For a sequence of length $N$, this repeats $2N$ times. Each wait triggers CPU–GPU IPC and forces the JavaScript thread to oscillate between idle and active states—an expensive pattern.

\textit{Advantages of a single, monolithic stream.}

We preserve the per-wavefront logic but perform multiple \texttt{beginComputePass} calls during command recording only, issuing one final \texttt{submit}. The GPU can execute the stream from start to finish without CPU stalls; driver validation and scheduling costs decrease significantly, and consecutive \texttt{dispatch}/\texttt{copy} commands smooth out DRAM traffic. Compressing $2N$ IPC events into one significantly reduces runtime, validating the effectiveness of batch submission.
\begin{figure}[htbp]
    \centering
    \includegraphics[width=0.9\linewidth]{1. 單一 CommandBuffer 示意圖.png}
    \caption{Single-CommandBuffer batch-submission workflow}
    \label{fig:scb-batch-workflow}
\end{figure}

Multiple wavefront \texttt{dispatchWorkgroups} calls are first recorded into one \texttt{CommandBuffer}; the host then sends the buffer to the GPU with a single \texttt{queue.submit()}, removing $2N$ rounds of IPC and scheduling overhead.

\subsubsection{Dynamic Uniform Offset — Cutting Constant-Update Overhead}
As illustrated in Figure~\ref{fig:dynamic-offset-layout}, per-wavefront constants \texttt{(len, diag, numGroups)} are stored consecutively in one uniform buffer, aligned to 256-byte blocks.

\begin{figure}[htbp]
    \centering
    \includegraphics[width=0.9\linewidth]{2. Dynamic Offset 佈局圖.png}
    \caption{Data layout for Dynamic Uniform Offsets}
    \label{fig:dynamic-offset-layout}
\end{figure}

WebGPU allows a 256-byte–aligned dynamic offset in \texttt{setBindGroup()} (W3C, 2024). At dispatch time, a dynamic offset selects the required block, so we avoid buffer churn even though \texttt{createBindGroup()} is still called. Validation overhead per call drops by roughly 20–30\% in practice (Google Chrome Developers, 2024).

\section{Summary}
Enabling all two browser-side optimizations simultaneously yields the performance results shown in Table~\ref{tab:opt_performance}. On an RTX 2070 Super, runtime drops from 466 s to 74 s—an 84 \% reduction (our measurements, 2025).
\begin{table}[h]
    \centering
    \setlength{\tabcolsep}{6pt}
    \renewcommand{\arraystretch}{1.4}
    \small
    \begin{tabularx}{\textwidth}{|X|X|X|X|}
        \hline
        Metric & Baseline & Optimized & Reduction \\
        \hline
        CPU$\leftrightarrow$GPU round-trips & $2N$ & 1 & $-99.999\%$ \\
        BindGroup entries & $2N \times \geq 10$ & $2N \times 7$ & $-30\%$ \\
        Execution time ($N = 100{,}000$) & 466 s & 74 s & $-84\%$ \\
        \hline
    \end{tabularx}
    \caption{Overview of performance gains from the two browser-side optimizations}
    \label{tab:opt_performance}
\end{table}
On the RTX 2070 Super, the WebGPU-Optimized version trails the CUDA implementation by only 19\% for a 100,000-base sequence, yet remains nearly three orders of magnitude faster than single-threaded CPU execution. This demonstrates that near-native GPU performance is achievable entirely within the browser sandbox.



	
\chapter{Results}
\section{Experimental Environment}
To ensure the reproducibility of our performance data, all WebGPU tests were run in Chrome~135.0.9049.114 (Google Chrome Developers, 2024). For comparability, the Apple M1 and Intel UHD 620 platforms were upgraded to the same browser version, and the relevant OS versions are listed alongside.


Table~\ref{tab:exp_env} summarises the three hardware setups and software stacks (NVIDIA, 2019; Apple, 2020; Intel, 2018) that serve as our baseline for later experiments.


\begin{table}[htbp]
  \centering
  \caption{Experimental environment}
  \label{tab:exp_env}
  \setlength{\tabcolsep}{8pt}
  \renewcommand{\arraystretch}{1.4}
  \begin{tabularx}{\textwidth}{@{}lX X X X@{}}
    \toprule
    Category & Parameter & RTX 2070 Super & Apple M1 GPU & Intel UHD 620 \\
    \midrule
    CPU      & Model                 & Ryzen 7 3700X        & Apple M1 (4P + 4E) & Core i5-8265U \\
    GPU      & SM / FP32 Peak        & 40 SM – 9.1 TFLOPS   & 8 Cores – 2.6 TFLOPS & 24 EU – 0.35 TFLOPS \\
    OS       & Version               & Ubuntu 24.04.2 LTS   & macOS 14.4           & Windows 11 22H2 \\
    Browser  & Version               & Chrome 135.0.9049.114 & same as above        & same as above \\
    CUDA drv & Version               & CUDA Toolkit 12.0 / Driver 550.54 & — & — \\
    \bottomrule
  \end{tabularx}
\end{table}

\section{Performance Data}
\subsection{RTX 2070 Super: Runtime and Speed-ups of Four Versions}
Wall-clock time $T(N)$ is defined as the elapsed time from the host call to the algorithm until the device returns the result, including GPU memory allocation and \texttt{queue.submit()}.

Under this definition, Table~\ref{tab:rtx_performance} lists the measured runtimes of C++, CUDA, WebGPU-Baseline (Init), and WebGPU-Optimized (Opt.) for four sequence lengths, together with their relative speed-ups:
\[
S_{X \leftarrow Y}(N) = \frac{T_Y(N)}{T_X(N)}
\]

\begin{table}[h]
    \centering
    \renewcommand{\arraystretch}{2}
    \setlength{\tabcolsep}{4pt}
    \small
    \begin{tabular}{|c|>{\centering\arraybackslash}p{2cm}|>{\centering\arraybackslash}p{2cm}|>{\centering\arraybackslash}p{2.2cm}|>{\centering\arraybackslash}p{2.2cm}|>{\centering\arraybackslash}p{2cm}|>{\centering\arraybackslash}p{2cm}|}
        \hline
        $N$ & CPU T (s) & CUDA T (s) & WGPU-Init (s) & WGPU-Opt. (s) & Opt./CPU & Opt./CUDA \\
        \hline
        $10^2$ & 0.00330 & 0.00229 & 0.135 & 0.020 & 0.165$\times$ & 0.11$\times$ \\
        $10^3$ & 0.327 & 0.0208 & 0.602 & 0.043 & 7.6$\times$ & 0.49$\times$ \\
        $10^4$ & 32.80 & 0.1908 & 21.83 & 0.346 & 94.8$\times$ & 0.55$\times$ \\
        $10^5$ & 3275.6 & 2.7696 & 466.8 & 3.299 & 993$\times$ & 0.94$\times$ \\
        \hline
    \end{tabular}
    \caption{Runtime and speed-ups of four versions on the RTX 2070 Super}
    \label{tab:rtx_performance}
\end{table}


\textit{Key finding:} WebGPU-Optimized sustains between 49\% and 94\% of CUDA performance, while delivering up to 993$\times$ acceleration over single-threaded CPU execution.

At $N = 10^2$, WebGPU-Opt incurs V8 start-up and IPC latency, reaching only 11\% of CUDA throughput. As sequence length increases, the use of Dynamic Uniform Offsets mitigates these bottlenecks. By $N = 10^5$, WebGPU-Opt achieves 84\% of CUDA performance with only a 0.53-second gap.

\begin{figure}[htbp]
    \centering
    \includegraphics[width=0.9\linewidth]{2070s-1.png}
    \caption{Speed-up of WebGPU-Baseline over CPU on the RTX 2070 Super}
    \label{fig:2070s-wgpu-baseline}
\end{figure}

\begin{figure}[htbp]
    \centering
    \includegraphics[width=0.9\linewidth]{2070s-3.png}
    \caption{Speed-up comparison of CUDA and WebGPU-Baseline (CPU = 1.0) on the RTX 2070 Super}
    \label{fig:2070s-cuda-vs-baseline}
\end{figure}

\begin{figure}[htbp]
    \centering
    \includegraphics[width=0.9\linewidth]{2070s-2.png}
    \caption{Speed-up of WebGPU-Optimized versus Baseline on the RTX 2070 Super}
    \label{fig:2070s-optimized-vs-baseline}
\end{figure}

\begin{figure}[htbp]
    \centering
    \includegraphics[width=0.9\linewidth]{2070s-4.png}
    \caption{Overall speed-up of the four versions (CPU / CUDA / Baseline / Optimized) on the RTX 2070 Super}
    \label{fig:2070s-overall-4versions}
\end{figure}

\subsection{Apple M1 and Intel UHD 620: Cross-Platform Performance}
Because these iGPUs cannot run CUDA, we measure WebGPU-Opt's pure acceleration over single-threaded CPU using:
\[
S_{\text{Opt} \leftarrow \text{CPU}}(N) = \frac{T_{\text{CPU}}(N)}{T_{\text{Opt}}(N)}
\]


\begin{table}[h]
    \centering
    \setlength{\tabcolsep}{4pt}
    \renewcommand{\arraystretch}{2}
    \small
    \begin{tabularx}{\textwidth}{|c
        |>{\centering\arraybackslash}X
        |>{\centering\arraybackslash}X
        |>{\centering\arraybackslash}X
        |>{\centering\arraybackslash}X
        |>{\centering\arraybackslash}X
        |>{\centering\arraybackslash}X|}
        \hline
        $N$ & M1 CPU (s) & M1 Opt. (s) & Opt./CPU & UHD CPU (s) & UHD Opt. (s) & Opt./CPU \\
        \hline
        $10^2$ & 0.00391 & 0.045 & 0.09$\times$ & 0.0101 & 0.136 & 0.07$\times$ \\
        $10^3$ & 0.308 & 0.034 & 9.1$\times$ & 0.936 & 0.234 & 4.0$\times$ \\
        $10^4$ & 31.38 & 0.272 & 115$\times$ & 95.51 & 1.524 & 62.7$\times$ \\
        $10^5$ & 3347.6 & 7.245 & 463$\times$ & 10851 & 48.79 & 222$\times$ \\
        \hline
    \end{tabularx}
    \caption{Acceleration of WebGPU-Optimized over CPU on Apple M1 and Intel UHD 620}
    \label{tab:cross_platform}
\end{table}


\begin{figure}[h]
    \centering
    \includegraphics[width=0.9\linewidth]{uhd620.png}
    \caption{WebGPU-Optimized speed-up over CPU on Intel UHD 620}
    \label{fig:uhd620}
\end{figure}

\begin{figure}[h]
    \centering
    \includegraphics[width=0.9\linewidth]{m1.png}
    \caption{WebGPU-Optimized speed-up over CPU on Apple M1}
    \label{fig:m1}
\end{figure}

\begin{figure}[h]
    \centering
    \includegraphics[width=0.9\linewidth]{比較不同硬體(RTX 2070 Super、M1、UHD 620)在 N=10⁵ 時的 WebGPU-Optimized 加速比.png}
    \caption{Comparison of WebGPU-Optimized speed-ups (CPU = 1.0) across three GPUs at $N=10^5$}
    \label{fig:cross-hw1}
\end{figure}
\clearpage

\section{Correctness Verification—Relative Log-Likelihood Error}
Using CUDA on the RTX 2070 Super as the golden standard, we compute the relative error:
\[
\varepsilon(N) = \frac{|LL_{\text{platform}}(N) - LL_{\text{CUDA}}(N)|}{|LL_{\text{CUDA}}(N)|} \times 100\%
\]

\begin{table}[h]
  \centering
  \caption{Relative Log-Likelihood error (vs.\ CUDA-2070 S) on each platform}
  \label{tab:likelihood_error}
  \setlength{\tabcolsep}{6pt}
  \renewcommand{\arraystretch}{1.7}
  \small
  \begin{tabularx}{\textwidth}{@{} X c c c c c @{}}
    \toprule
    Platform / $N$      & $10^2$           & $10^3$           & $10^4$           & $10^5$           & Max.\ error    \\
    \midrule
    WGPU-Opt -- 2070 S  & $2.5\times10^{-4}\,\%$ & $1.3\times10^{-5}\,\%$ & $2.2\times10^{-4}\,\%$ & $3.8\times10^{-4}\,\%$ & $3.8\times10^{-4}\,\%$ \\
    WGPU-Opt -- M1      & $2.8\times10^{-4}\,\%$ & $1.5\times10^{-5}\,\%$ & $2.2\times10^{-4}\,\%$ & $3.8\times10^{-4}\,\%$ & $3.8\times10^{-4}\,\%$ \\
    WGPU-Opt -- UHD 620 & $2.5\times10^{-4}\,\%$ & $1.3\times10^{-5}\,\%$ & $2.2\times10^{-4}\,\%$ & $3.8\times10^{-4}\,\%$ & $3.8\times10^{-4}\,\%$ \\
    \bottomrule
  \end{tabularx}
\end{table}


\section{Summary}
Overall, WebGPU-Optimized achieves up to 88\% of CUDA performance on the RTX 2070 Super while retaining a three-order-of-magnitude advantage over single-threaded CPU execution.

On Apple M1 and Intel UHD 620, the same WGSL shader still achieves 4–463$\times$ acceleration, demonstrating that our two optimizations are portable and do not depend on vendor-specific extensions.

Across all platforms, relative Log-Likelihood errors remain below $4 \times 10^{-4}$, ensuring both high performance and numerical correctness.


\chapter{Discussion}
\section{Performance Differences and Bottlenecks}
The experiments reveal that—even on the RTX 2070 Super—our WebGPU-Optimized version still lags CUDA by 12\%–88\%. On the Apple M1 and Intel UHD 620, the optimized shader achieves dozens- to hundreds-fold speed-ups over single-threaded CPU code, yet its absolute runtime remains higher than CUDA's. Hence, the bottlenecks lie not in the algorithmic flow but in the interaction between micro-architecture and API design. We therefore analyse three root causes: the absence of special-function units (SFUs), differences in cache paths, and resource-binding overhead.

\subsection{Impact of Missing SFUs on \texttt{log}/\texttt{exp} Throughput}
As shown in Figure~\ref{fig:log_exp_pipeline}, every streaming multiprocessor (SM) in post-Volta CUDA GPUs contains 32 special-function units (SFUs) that complete an entire warp's \texttt{log}/\texttt{exp} operations in four cycles.

To guarantee consistent semantics across NVIDIA, AMD, Intel, and Apple devices, WebGPU must decompose each \texttt{log}/\texttt{exp} into mantissa/exponent extraction, LUT approximation, and two fused-multiply-add (FMA) steps for sixth-order polynomial correction—resulting in 11–12 cycles per call.

\begin{figure}[h]
    \caption{Latency comparison between CUDA SFU and WebGPU software \texttt{log}/\texttt{exp} pipeline}
    \label{fig:log_exp_pipeline}
\end{figure}

CUDA completes these in four hardware cycles, while WebGPU requires:
mantissa split (1 ALU cycle) $\rightarrow$ LUT interpolation (4 cycles) $\rightarrow$ FMA correction (4 cycles) $\rightarrow$ logarithm scaling and write-back (2–3 cycles). At 1.7 GHz, this translates to a theoretical peak of 320 vs.\ 170 ops per cycle, respectively.

Each DP cell in Pair-HMM invokes around 30 \texttt{log}/\texttt{exp} calls. At $N=10^5$, this amounts to $3 \times 10^{11}$ total calls. Based on theoretical throughput, CUDA takes about 0.59 s, while WebGPU requires at least 1.01 s—a baseline gap of 0.42 s. When factoring in wavefront dependency and 65\% thread utilisation, the effective difference ranges from 0.25–0.30 s, or roughly 45–55\% of the total gap. Given that complexity scales with $N^2$, this latency becomes a dominant contributor at large $N$.

\subsection{Cache-Policy Gap: 32 KB L1 Hits vs. Storage-Path Bypass}
CUDA can issue \texttt{ld.global.ca/cg} to cache read-only data in the 32 KB L1 cache or 64 KB sector cache. On TU104, a single L1 hit costs approximately 20 ns. The DP rows in Pair-HMM are memory-contiguous and thus cache-friendly.

In contrast, Dawn maps WGSL's \texttt{var<storage>} to \texttt{ld.global.cg} and \texttt{st.global.cg}, bypassing L1 to preserve cross-workgroup visibility. Even after relocating the 336 B transition and emission tables into \texttt{var<workgroup>}, DP rows are still read from DRAM.

Consequently, the same 15 global reads take about 0.30 μs on CUDA but 1.2–1.5 μs on WebGPU—a significant latency tier difference.

\subsection{API Overhead: Pointer Rotation vs. BindGroup Reconstruction}
In CUDA, the host simply rotates three pointers to swap \texttt{prev}, \texttt{curr}, and \texttt{new} with no allocation or validation overhead.

WebGPU, however, requires a new \texttt{createBindGroup()} call whenever the binding changes, due to descriptor immutability. Each such call traverses the V8 $\rightarrow$ Blink $\rightarrow$ Dawn $\rightarrow$ Driver stack, taking 5–15 μs. At $N = 10^5$, the $2N$ anti-diagonals trigger 200,000 such reconstructions—accumulating several seconds of delay.

While our use of Dynamic Uniform Offsets eliminates repeated uniform bindings, the three DP buffers remain mutable and thus still necessitate $2N$ BindGroup creations.

\subsection{Quadratic Amplification and Energy Implications}
Because Pair-HMM's computational complexity scales with $N^2$, even minor latencies are quadratically amplified.

At $N = 100$, SFU absence is masked by cache hits. At $N = 100,000$, it imposes a floor of approximately 1.0 s in runtime. With additional latency from DRAM traffic and BindGroup overhead, WebGPU-Optimized completes in 3.3 s, versus 2.77 s for CUDA.

These results confirm that architectural and API differences—more so than algorithmic inefficiencies—are the principal reasons WebGPU cannot yet match CUDA's raw performance.

\section{Cross-Hardware Performance}
\subsection{Apple M1: Pros and Cons of a Unified-Memory Architecture (UMA)}
The Apple M1's UMA design shares 8 GB of LPDDR4X between CPU and GPU, eliminating the need for discrete memory transfers. For moderate values of $N$, \texttt{copyBufferToBuffer()} is reduced to pointer offset adjustment rather than true DMA, resulting in lower WebGPU start-up latency than on discrete GPUs.

At large $N$, however, bandwidth contention between CPU and GPU becomes visible, and the M1 remains approximately 2.2$\times$ slower than the RTX 2070 Super.

Nonetheless, WebGPU-Optimized achieves a 463$\times$ speed-up over CPU execution, confirming that Dynamic Uniform Offsets effectively hide latency even under unified memory.

\subsection{Intel UHD 620: Driver Maturity and Scheduling Strategy}
The UHD 620 lacks hardware SFUs and contains only 24 execution units (EUs), which makes intensive \texttt{log}/\texttt{exp} workloads more costly. Chrome–Dawn–DX12 submission still serialises commands via a submit-fence pattern, causing idle CPU periods at small $N$.

Its 768 KB L3 cache is easily thrashed when multiple storage buffers are interleaved, resulting in frequent L2 cache misses. Despite this, our optimisations deliver a 222$\times$ speed-up at $N = 100,000$.

These results confirm that the proposed WebGPU optimisations are vendor-agnostic and capable of improving performance even on low-end integrated GPUs, underscoring WebGPU's potential as a cross-platform acceleration framework.



\chapter{Future Work}

This study demonstrates that the two WebGPU-oriented optimizations—\emph{single-CommandBuffer batch submission} and \emph{Dynamic Uniform Offsets}—can reduce the browser-side runtime of the Pair-HMM Forward algorithm to within a constant factor of native CUDA. While this level of performance already supports online demonstrations and interactive teaching, further improvements remain desirable for large-scale clinical pipelines and cloud back-end deployments. Accordingly, we outline three promising directions—at the API, algorithmic, and ecosystem levels—and explain why each represents a logical next step.

\section{Closing the Double-Precision Gap}

In GPU-accelerated scientific computing, \emph{FP64} support is often critical for maintaining numerical stability. WebGPU currently guarantees only \emph{FP32}; even on hardware with native double-precision units (e.g., RTX 40-series or Apple M2 Max), WGSL lacks an official \texttt{f64} type (W3C, 2024; NVIDIA, 2023; Apple, 2023). For the Pair-HMM Forward algorithm, 32-bit floating-point precision typically keeps relative error below $10^{-5}$. However, very long reads or the accumulation of extremely small probabilities can still lead to underflow and numerical degradation.

Future work could address this issue through two strategies: (i) extending the WebGPU specification to support \texttt{f64} types, or (ii) applying \emph{mixed-precision} techniques to selectively elevate precision for sensitive calculations inside the shader. Either approach would expand WebGPU's utility for high-sensitivity genomic analyses.

\section{Hybrid Acceleration with WASM + SIMD and WebGPU}

Although WebGPU offers high throughput for large workloads, its fixed API and driver overheads become bottlenecks for short reads or fragmented input. A practical solution is to combine \emph{WebAssembly (WASM)} with 128-bit SIMD as a front-end processing layer (MDN Web Docs, 2023):

\begin{itemize}
    \item For short sequences (e.g., $N < 512$), WASM + SIMD can execute the entire algorithm on the CPU, avoiding GPU cold-start latency (Google Chrome Developers, 2024).
    \item For longer sequences, WebGPU processes thousands of reads per dispatch, leveraging its parallel processing strengths.
\end{itemize}

This hybrid approach improves front-end responsiveness and mitigates WebGPU's latency and lack of hardware SFUs on small-scale workloads. It encourages cooperative use of CPU and GPU resources in the browser environment.

\section{Community Standardisation and an Open-Source Ecosystem}

Currently, no standardised \emph{browser-native GPU benchmark suite} exists for bioinformatics. Packaging the proposed WGSL shader and JavaScript harness as an open-source NPM module would allow browser and GPU vendors to benchmark cache behavior under realistic conditions. It would also enable the research community to port other wavefront-based algorithms—such as Smith–Waterman, Needleman–Wunsch, and BWA-MEM—to the Web platform.

Releasing such a toolkit early could promote community-driven discussion of API best practices, increasing interoperability and reducing the risk of breaking changes as the WebGPU specification evolves (W3C, 2024). A shared ecosystem would also accelerate adoption of high-performance browser-based genomic tools.

\section*{Conclusion}

Future research directions—namely, enabling double-precision support, introducing hybrid WASM–WebGPU execution models, and developing an open benchmark ecosystem—offer clear paths to improving the performance, portability, and scientific rigor of WebGPU-based bioinformatics pipelines. These efforts will further democratise high-performance computing in genomics and enhance the long-term utility of the optimizations proposed in this work.






\chapter{Conclusion}

This work presents the first complete browser-side implementation of the Pair-HMM Forward algorithm using WebGPU, filling a gap left by earlier CUDA-centric studies (Banerjee \emph{et al}., 2017; Schmidt \emph{et al}., 2024). We systematically evaluate four implementations—C++, CUDA, WebGPU-Baseline, and WebGPU-Optimized—across three heterogeneous GPUs (RTX 2070 Super, Apple M1, and Intel UHD 620), analysing both performance and numerical correctness.

\section{Core Contributions}

\begin{enumerate}
    \item \emph{Two complementary browser-side optimizations.} \\
    The proposed techniques—single-CommandBuffer batch submission and Dynamic Uniform Offsets reduce CPU–GPU round-trips, minimise BindGroup reconstruction overhead. For a sequence of length $N = 10^{5}$ on the RTX 2070 Super, these optimizations cut runtime from 467 s (Baseline) to 3.3 s, achieving 84 \% of CUDA performance.

    \item \emph{Cross-device validation.} \\
    The same WGSL shader delivers 4–222$\times$ speed-up on Intel UHD 620 (Intel, 2018) and 9–463$\times$ on Apple M1 (Apple, 2020), confirming that the optimizations are vendor-agnostic: any browser supporting WebGPU can offer GPU acceleration without native driver installation.

    \item \emph{A reproducible workflow for porting bioinformatics DP algorithms to the browser.} \\
    We provide WGSL implementations and tuning strategies that address known WebGPU bottlenecks, creating a template for porting wavefront-based algorithms such as Smith–Waterman and Needleman–Wunsch to the web (Ghosh \emph{et al}., 2018).
\end{enumerate}

\section{Academic and Industrial Impact}

WebGPU is not merely a replacement for installing CUDA SDKs or provisioning cloud GPUs; it enables real-time computation within the browser sandbox, with zero driver installation and local data residency (W3C, 2024). Researchers can now perform Pair-HMM likelihood estimation on laptops and iGPU systems while retaining full data control—lowering the threshold for classroom use, clinical front-ends, and interactive open-science platforms (Google Chrome Developers, 2024).

In summary, the model demonstrated—driver-free, cross-hardware, and fully on-device—illustrates a concrete path for browser-native scientific GPU computing. As browser APIs and GPU hardware evolve, we anticipate many genomics applications will become fully web-executable within the next three to five years, further democratising high-performance bioinformatics and accelerating digital transformation in the biomedical domain.




\chapter{References}
\begin{enumerate}
  \item Liu, Y., Schrinner, S., \& others. (2021). \emph{GPU Acceleration in Genomics: A Comprehensive Survey}. \textit{Briefings in Bioinformatics}, 22(5), bbab042. \url{https://doi.org/10.1093/bib/bbab042}

  \item W3C. (2024). \emph{WebGPU Specification}. W3C Recommendation. \url{https://www.w3.org/TR/webgpu/}

  \item Chou, Y.-C.\ (2024). \emph{Pair-HMM Forward: Reference \& GPU-Accelerated Implementations}. GitHub repository. \url{https://github.com/yuchen0620/ChouYuchen-master-thesis}

  \item Durbin, R., Eddy, S. R., Krogh, A., \& Mitchison, G. (1998). \emph{Biological Sequence Analysis: Probabilistic Models of Proteins and Nucleic Acids}. Cambridge University Press.

  \item Mardis, E. R. (2017). \emph{DNA Sequencing Technologies: 2006–2016}. \textit{Nature Protocols}, 12(2), 213–218. \url{https://doi.org/10.1038/nprot.2016.182}

  \item McKenna, A., et al. (2010). \emph{The Genome Analysis Toolkit: A MapReduce Framework for Analyzing Next-Generation DNA Sequencing Data}. \textit{Genome Research}, 20(9), 1297–1303. \url{https://doi.org/10.1101/gr.107524.110}

  \item Li, H., et al. (2009). \emph{The Sequence Alignment/Map Format and SAMtools}. \textit{Bioinformatics}, 25(16), 2078–2079. \url{https://doi.org/10.1093/bioinformatics/btp352}

  \item Vasimuddin, M., Misra, S., Li, H., \& Aluru, S. (2019). \emph{Efficient Architecture-Aware Acceleration of BWA-MEM for Multicore Systems}. \textit{Proceedings of IEEE IPDPS 2019}, 314–324. \url{https://doi.org/10.1109/IPDPS.2019.00041}

  \item Banerjee, S. S., et al. (2017). \emph{Hardware Acceleration of the Pair-HMM Algorithm for DNA Variant Calling}. \textit{Proc. 27th FPL}, 165–172. \url{https://doi.org/10.23919/FPL.2017.8056826}

  \item Google Chrome Developers. (2024). \emph{WebGPU Now Available in Chrome}. Chromium Blog. \url{https://blog.chromium.org/2024/05/webgpu-now-available.html}

  \item NVIDIA Corporation. (2023). \emph{CUDA C++ Programming Guide v12.4}. \url{https://docs.nvidia.com/cuda/cuda-c-programming-guide}

  \item Ghosh, P., et al. (2018). \emph{Web3DMol: Interactive Protein Structure Visualization Based on WebGL}. \textit{Bioinformatics}, 34(13), 2275–2277. \url{https://doi.org/10.1093/bioinformatics/bty534}

  \item Google Chrome Team. (2024). \emph{Chrome's 2024 Recap for Devs: Re-imagining the Web with AI}. Chrome for Developers Blog. \url{https://developer.chrome.com/blog/chrome-2024-recap}

  \item Illumina. (2024). \emph{NovaSeq X Series Reagent Kits – Specifications}. \url{https://www.illumina.com/systems/sequencing-platforms/novaseq-x-plus/specifications.html}

  \item Jones, B. (2023). \emph{Toji.dev Blog Series: WebGPU Best Practices}. \url{https://toji.dev/webgpu-best-practices/}

  \item Klöckner, A., Pinto, N., Lee, Y., Catanzaro, B., Ivanov, P., \& Fasih, A. (2012). \emph{PyCUDA and PyOpenCL: A Scripting-Based Approach to GPU Run-Time Code Generation}. \textit{Parallel Computing}, 38(3), 157–174. \url{https://doi.org/10.1016/j.parco.2011.09.001}

  \item Krampis, K., Booth, T., Chapman, B., et al. (2012). \emph{Cloud BioLinux: Pre-configured and On-Demand Bioinformatics Computing for the Genomics Community}. \textit{BMC Bioinformatics}, 13, 42. \url{https://doi.org/10.1186/1471-2105-13-42}

  \item Langmead, B., Trapnell, C., Pop, M., \& Salzberg, S. L. (2009). \emph{Ultrafast and Memory-Efficient Alignment of Short DNA Sequences to the Human Genome}. \textit{Genome Biology}, 10(3), R25. \url{https://doi.org/10.1186/gb-2009-10-3-r25}

  \item Li, H., \& Durbin, R. (2010). \emph{Fast and Accurate Long-Read Alignment with Burrows–Wheeler Transform}. \textit{Bioinformatics}, 26(5), 589–595. \url{https://doi.org/10.1093/bioinformatics/btq698}

  \item Liu, Y., Wirawan, A., \& Schmidt, B. (2013). \emph{CUDASW++ 3.0: Accelerating Smith–Waterman Protein Database Search by Coupling CPU and GPU SIMD Instructions}. \textit{BMC Bioinformatics}, 14, 117. \url{https://doi.org/10.1186/1471-2105-14-117}

  \item MDN Web Docs. (2025). \emph{WebGPU API}. \url{https://developer.mozilla.org/en-US/docs/Web/API/WebGPU_API}

  \item Schmidt, B., et al. (2024). \emph{gpuPairHMM: High-Speed Pair-HMM Forward Algorithm for DNA Variant Calling on GPUs}. arXiv preprint, arXiv:2411.11547. \url{https://arxiv.org/abs/2411.11547}

  \item Stone, J. E., Gohara, D., \& Shi, G. (2010). \emph{OpenCL: A Parallel Programming Standard for Heterogeneous Computing Systems}. \textit{Computing in Science \& Engineering}, 12(3), 66–73. \url{https://doi.org/10.1109/MCSE.2010.69}

  \item TensorFlow.js Team. (2024). \emph{WebGPU Backend for TensorFlow.js}. \url{https://www.tensorflow.org/js/guide/webgpu}

  \item W3C. (2024). \emph{WebGPU Specification: Candidate Recommendation Snapshot}. \url{https://www.w3.org/TR/2024/CR-webgpu-20241219/}

  \item Illumina. (2024). \emph{NovaSeq X Plus Specification Sheet}. Technical white paper. \url{https://www.illumina.com/content/dam/illumina-marketing/documents/products/datasheets/novaseq-x-plus-specifications.pdf}

  \item Ferragina, P., \& Manzini, G. (2000). \emph{Opportunistic Data Structures with Applications}. \textit{Proc. 41st IEEE FOCS}, 390–398. \url{https://doi.org/10.1109/SFCS.2000.892127}

  \item Xenova. (2024). \emph{Transformers.js: Running Transformer Models Directly in the Browser}. GitHub repository. \url{https://github.com/xenova/transformers.js}

  \item NVIDIA. (2019). \emph{GeForce RTX 2070 SUPER Founders Edition Specifications}. Product brief. \url{https://www.nvidia.com/en-us/geforce/graphics-cards/rtx-2070-super/specs}

  \item Apple. (2020). \emph{Apple M1 Chip — Technical Overview}. Apple Developer Documentation. \url{https://developer.apple.com/documentation/apple_silicon/apple_m1}

  \item Intel. (2018). \emph{Intel UHD Graphics 620 — Product Specifications}. Intel ARK. \url{https://ark.intel.com/content/www/us/en/ark/products/126789}

  \item Apple. (2023). \emph{Apple M2 Max Chip — Technical Overview}. Apple Developer Documentation. \url{https://developer.apple.com/documentation/apple_silicon/apple_m2_max}

  \item MDN Web Docs. (2023). \emph{WebAssembly SIMD}. \url{https://developer.mozilla.org/en-US/docs/WebAssembly/SIMD}
\end{enumerate}


\newpage
\AddToContents{Bibliography}
\printbibliography
\end{document}